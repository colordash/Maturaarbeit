\chapter*{Vorwort}

Da wir beide in unserer Freizeit viel mit Drohnen gespielt und uns intensiv mit ihnen befasst haben, stellte sich irgendwann die unabdingbare Frage: \textit{Wie funktionieren Drohnen eigentlich, und wie sind sie aufgebaut?} Um diese Frage zu klären, haben wir uns entschieden, unsere Maturaarbeit diesem faszinierenden Thema zu widmen. Drohnen vereinen zahlreiche Technologien, von moderner Mechanik über ausgeklügelte Elektronik bis hin zu hochentwickelter Software, was sie zu einem spannenden Forschungsfeld macht. Wir möchten sowohl die Hardware als auch die Software einer Drohne genauer untersuchen, um ein tieferes Verständnis für die Funktionsweise dieser Geräte zu entwickeln.

Anfangs hatten wir geplant, das Steuerungsprogramm der Drohne komplett selbst zu schreiben. Doch im Laufe unserer Recherchen haben wir schnell erkannt, dass dies den Rahmen einer Maturaarbeit sprengen würde. Zudem haben wir erkannt, dass dieses Thema von beinahe allen Personen unterschätzt wird und die Entwicklung einer solchen Software ist ein komplexes Unterfangen, das weitreichende Kenntnisse in Programmierung, Steuerungstechnik und Signalverarbeitung erfordert. Stattdessen entschieden wir uns, uns auf den Aufbau der Drohne und die Analyse bestehender Software zu konzentrieren. Wir wollten durch die Untersuchung spezifischer Funktionen und durch gezielte Anpassungen des Drohnencodes die Funktionsweise besser nachvollziehen und gleichzeitig praxisnahe Erfahrungen sammeln.

Unser Ziel ist es, die Technologie einer Drohne umfassend zu analysieren – von der Hardware, die für ihre Bewegungen verantwortlich ist, bis hin zum Code, der für die präzise Steuerung und autonome Funktionalität sorgt. So hoffen wir, ein tieferes Verständnis in diese faszinierende Technologie zu gewinnen.

Abschliessend möchten wir anmerken, dass diese Arbeit auch ein Stück Leidenschaft widerspiegelt. Die Begeisterung, die wir bei der Auseinandersetzung mit Drohnen erleben durften, hat uns durch die anspruchsvollen Phasen des Projekts getragen. Wir sind gespannt, welche Erkenntnisse wir auf unserem Weg, den wir dank unseres Sponsors, dem Altersheim Abendruh \cite{Altersheim}, gehen können, noch gewinnen werden.