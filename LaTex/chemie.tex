\chapter{Chemie}\label{sec:chemie}
Mit dem alleinigen zusammenbau einer Drohne ist eine Fliegende drohne noch längst nicht geschaft. Als nächstes muss man die Kommunikation zwischen Fernbedienung und Drohne sicherstellen.  



\section{Der Sender (Fernbedienung)}
Es gibt mehrere Packages. Exemplarisch ist hier das <<mhchem>>\cite{mhchem}
Package vorgestellt:


\section{Der Empfänger} \label{sec:chemie-verwendung}
Am besten lesen Sie die Dokumention des Packages, das ist voll mit vielen
Beispielen. Hier sind einige daraus:

\ce{H2O}

\ce{Na+} und \ce{Cl-}

\ce{^{227}_{90}Th}

Der entsprechende Code ist in \autoref{fig:chemie} zu finden.

\begin{figure}[ht]
\centering
\begin{minipage}{0.8\textwidth}
\begin{verbatim}
\ce{H2O}

\ce{Na+} und \ce{Cl-}

\ce{^{227}_{90}Th}
\end{verbatim}
\end{minipage}
\caption{\LaTeX{}-Code, der die chemischen Formeln in
	\autoref{sec:chemie-verwendung}
erzeugt.}
\label{fig:chemie}
\end{figure}
