\documentclass{article}
\usepackage{listings}
\usepackage{xcolor}
\usepackage[most]{tcolorbox}

% Box Stil für den Code-Block
\newtcblisting{codebox}{
	colframe=gray!75!black,  % Rahmenfarbe
	colback=black!90,        % Hintergrundfarbe
	coltitle=white,          % Farbe des Titels
	listing only,
	listing options={
		basicstyle=\ttfamily\color{white}, % Schriftstil und -farbe
		keywordstyle=\color{blue},         % Keywords-Farbe
		commentstyle=\color{green},        % Kommentare-Farbe
		stringstyle=\color{red},           % Strings-Farbe
		numberstyle=\tiny\color{gray},     % Zeilennummern-Farbe
		numbers=left,                      % Zeilennummern anzeigen
		frame=single,                      % Rahmen um den Code
		breaklines=true,                   % Zeilenumbruch bei langen Zeilen
		tabsize=4                          % Tabulatorgröße
	},
	title=latex,             % Titel oben links (kann angepasst werden)
	fonttitle=\bfseries,
	left=5mm,
	enhanced,
	overlay={
		% Textfeld rechts oben mit "Code kopieren"
		\node[anchor=north east, white, font=\bfseries] at (frame.north east) {Code kopieren};
	}
}

\begin{document}
	
	Wir machen uns auf den richtigen Weg. Wir haben uns in die Benutzung von WSL vertieft und damit erfolgreich den Source Code des Projektes in eine Hex-Datei kompiliert. Zudem haben wir gelernt, \texttt{git} in der Kommandozeile zu verwenden. Nicolas hat sich mit der Benutzung und Gestaltung von LaTeX auseinandergesetzt. Wir stehen noch vor dem Problem, dass es Schwierigkeiten gibt, die Hex-Datei über Betaflight Configurator auf die Drohne zu flashen.
	
	Es gab jedoch positive Erfolge beim Auslesen von Sensordaten und der Benutzung der Motoren mit Betaflight Configurator.
	
	\textbf{Hex Datei kompilieren:}
	
	\begin{codebox}
		sudo apt update && sudo apt upgrade -y
		sudo apt install python curl -y
		sudo apt install git
		
		# Öffne den Ordner des Projektes und mache ein Git-Repository daraus oder klone es direkt von der Homepage:
		git clone https://www.github.com/betaflight/betaflight [foldername]
		
		# Herunterladen der Build-Umgebung. Das Ganze dauert eine Weile. Zur Überwachung gibt es den Befehl:
		htop
		sudo apt install build-essential
		sudo make arm_sdk_install
		
		# Erstellen der Hex-Datei:
		make configs 
		make [targetname]   # Beispiel wäre STM32F7X2
	\end{codebox}
	
	\textbf{Git zum Laufen bringen:}
	
	\begin{codebox}
		# Git herunterladen und updaten:
		sudo apt install git 
		sudo apt install keychain
		git config --global user.name "DeinName"
		git config --global user.email "DeineE-Mail"
		
		# SSH-Key einrichten:
		ssh-keygen -t ed25519 -C [DeineEMail]
		cat ~/.ssh/id_ed25519.pub  # Kopieren und in GitHub einfügen
		eval `keychain --eval --agents ssh id_ed25519`  # In [nano ~/.bashrc] einfügen
		sudo reboot 
		
		# Repository herunterladen:
		git clone git@github.com:colordash/Drohne_Project.git
	\end{codebox}
	
\end{document}
