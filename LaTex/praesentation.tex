\chapter{Präsentation}\label{sec:praesentation}


\section{Formale Vorgaben}
\begin{itemize}
	\item Die Präsentation dauert 15 Minuten ($\pm$ 1 Minute).
	\item Die Slides sind nummeriert (für Rückfragen danach).
	\item Die Schrift und Grafiken sind leserlich, konstrastreich und
		ansprechend.
	\item Diagramme sind mit allen nötigen Beschriftungen versehen, wie Achsen,
		Masseinheiten, Skala.
	\item Titelslide mit Titel der Maturaarbeit und Name des Autors
	\item Die Präsentation endet mit einem Slide, mit dem sich der Vortragende
		für die Aufmerksamkeit des Publikums bedankt.
\end{itemize}

\section{Tips und Tricks}
\begin{itemize}
	\item Dunkle Schrift auf hellem Grund ergibt den besten Kontrast und ist am
		einfachsten zu lesen\footnote{Kleinere Blende wegen mehr Licht.}. Wobei
		das vor allem für Hellraumprojektoren und Beamer wichtig ist, bei
		Bildschirmen hat man diesbezüglich mehr Freiheiten und kann auch mal
		«Darkmode» wagen.
	\item Pro Slide rechnet man mit etwa einer Minute.
	\item Überzählige Slides nicht löschen sondern nach dem «offiziell letzen»
		Slide platzieren. Sehr gut möglich, dass genau dazu Fragen gestellt
		werden.
	\item Nur Stichwörter auf den Slides, keine ganzen Sätze (ausser wichtige
		Zitate vielleicht). Slides sollen den Vortragenden und die Zuhörenden
		führen, nicht vorwegnehmen, was der Vortragende erzählen möchte.
	\item Wo möglich, mit Grafiken arbeiten und diese erklären. Das
		fesselt das Publikum eher als Text.
	\item Für Live-Demos nach Möglichkeit eine Video-Aufzeichnung als Backup
		bereithalten.
	\item Übergangsanimationen gar nicht oder nur sehr dezent einsetzen. Braucht
		Zeit und lenkt ab. Ausser es lässt sich mit dem Inhalt verbinden.
	\item Heben Sie in der Präsentation Ihre Eigenleistung hervor. 
	\item Sprechen Sie zum Publikum (und nicht zum Bildschirm).
	\item Kontrollieren Sie einmal am Anfang, dass auf dem
		Präsentationsbildschirm auch das erscheint, was erscheinen soll. Danach
		dürfen Sie sich darauf verlassen und ins Publikum schauen.
\end{itemize}


