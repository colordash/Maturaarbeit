\chapter{Einleitung}
\section{Themenvorstellung}
Drohnen haben in den letzten Jahren grosse Fortschritte gemacht und finden Anwendung in unterschiedlichsten Bereichen wie Logistik, Vermessung und Unterhaltung. Für die vielfältigen Einsatzmöglichkeiten ist das Verständnis der Funktionsweise und des Aufbaus einer Drohne von zentraler Bedeutung, nicht zuletzt für die Weiterentwicklung der Technologie. Insbesondere die Entwicklung von Steuermechanismen, wie zum Beispiel eines präzisen Landeschalters, stellt eine grundlegende Herausforderung dar, die praktische und technische Kompetenzen erfordert.
	
\section{Forschungsstand und Problemstellung}
Die Grundlagen der Drohnentechnologie – darunter Aerodynamik, Steuerung und Energiemanagement – sind in gut dokumentiert. Ebenso gibt es zahlreiche technische Ansätze zur Automatisierung einzelner Funktionen wie Start und Navigation. Doch die Frage, wie komplex diese Technologie wirklich ist und wie ein Schalter zur automatisierten Landung programmiert sowie in die bestehende Steuerung integriert werden kann, bildet den Schwerpunkt dieser Arbeit.
	
\section{Ziele der Arbeit}
Das Ziel dieser Arbeit ist es, die Funktionsweise und den Aufbau einer Drohne zu analysieren und einen funktionalen Landeschalter zu entwickeln. Dies umfasst die theoretische Untersuchung der technischen Grundlagen sowie die praktische Umsetzung und Programmierung des Schalters. Durch die Kombination von Theorie und Praxis soll ein besseres Verständnis für die zentralen Komponenten und Steuermechanismen einer Drohne vermittelt werden.
